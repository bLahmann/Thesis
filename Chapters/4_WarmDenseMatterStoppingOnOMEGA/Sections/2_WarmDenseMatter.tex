Warm Dense Matter (WDM) can be understood as the state of matter between cold solids and hot plasmas. There are often two dimensionless parameters used to classify the WDM regime. The first is the degeneracy parameter:
%
\begin{equation}
    \theta \equiv \frac{k_BT_e}{E_F}
\end{equation}
%
where $k_BT_e$ is the electron temperature and $E_F$ is the electron Fermi energy given by:
%
\begin{equation}
    E_F \equiv \frac{\hbar^2}{2m_e}\left(3\pi^2 n_e\right)^{2/3} 
\end{equation}

This parameter speaks to whether or not the electrons of a material are Fermi degenerate. When $\theta \gg 1$ the state is considered to be non-degenerate. In this regime, the distribution of electrons is described by a Maxwellian given by the plasma temperature. 

On the other hand, when $\theta \ll 1$, the material is said to be electron degenerate. In this state, the distribution of electrons is dictated by the Pauli Exclusion Principle. Electrons will (on average) have higher energy states than would be described by a Maxwellian. This is because the the material is so tightly bound that most of the electrons are in their lowest possible energy states. Since the Pauli Exclusion Principle disallows two electron from occupying the same state, several of them must be elevated in energy to exist within the material. 

The other parameter of interest is the coupling parameter:
%
\begin{equation}
    \Gamma_e = \frac{e^2}{a\left(k_BT_e + E_F\right)}
\end{equation}
%
where $a$ is the Wigner-Seitz radius given by:
%
\begin{equation*}
    a \equiv \left(\frac{3}{4\pi n_e}\right)^{1/3}
\end{equation*}

The coupling parameter is a ratio between the Coulomb potential energy and the electron thermal energy. When $\Gamma_e \ll 1$ the plasma is considered to be weakly coupled. This describes plasmas that are hot and diffuse and are dominated by long range collective electrostatic effects.

When $\Gamma_e \gg 1$ the material is said to be strongly coupled. Here binary collisions dominate over long ranged electrostatic effects. These materials are cold and dense.

WDM describes materials that are moderately degenerate ($\theta \sim 1$) and moderately coupled ($\Gamma_e\sim1$). This regime is interesting because few models have been made to describe it. More importantly, the cold-layer of ICF implosions exists in this regime during the burn phase of the implosion. 