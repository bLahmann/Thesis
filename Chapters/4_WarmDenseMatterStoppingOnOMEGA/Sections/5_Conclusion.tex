We have successfully developed and demonstrated a platform for generating and characterizing WDM plasma on the OMEGA laser facility. The WDM targets were elevated to temperatures just below 10 eV and achieved partial ionization. The stopping power of these targets were measured and shown to have increases relative to cold matter stopping when the temperatures and ionization are high enough.

Several pitfalls were identified in the development of this platform. The Zn Thomson Scattering foils must be truncated such that they never overhang the shield cone protecting the spectrometer. Additionally, no material that emits x-rays in the range of 8-9 keV (i.e Ni, Cu, Zn) should be used in the experiment. Finally, excellent improvement was demonstrated in the Thomson Scattering data using 3D printed cones covered in 75 $\mu$m of Ta. Lower Z materials may prove better due to lower energy L emissions, however, more thickness would be required to attenuate x-rays. 

Future work should consider Li targets if considered feasible from a manufacturing standpoint. Work should also be done to improve the efficiency of the heating drive to push to higher temperatures and ionization states.