\section*{Thesis Overview}
\addcontentsline{toc}{chapter}{Thesis Overview}

This thesis includes a variety of different works done by the author. The work is all loosely connected in that the projects all serve to increase the overall understanding of the National Inertial Confinement Fusion Program. They are also all loosely connected in the sense that they all seek to better understand some aspect of implosion areal densities ($\rho R)$. Despite these connections, the projects disparate enough to deserve separate chapters as they all were performed on unique experimental facilities (The National Ignition Facility, the Z Facility, and OMEGA Laser Facility) and all were independently financed. The format of this thesis is as follows:

\textbf{Chapter 1: Introduction} 

This Chapter serves to motivate nuclear fusion in general as an energy source of the future worth investigation. It later goes into the physics of nuclear fusion to explain basic terminology and design motivations. The basics of Inertial Confinement Fusion (ICF) and later Magnetized Liner Inertial Fusion (MagLIF) are explained with the importance of $\rho R$ being stressed. Finally, the Chapter ends with a brief introduction to the four experimental facilities used throughout this work.

\textbf{Chapter 2: Secondary DT Neutron Measurements on the NIF}

This Chapter summarizes all of the work performed on the National Ignition Facility (NIF). It describes the physics of secondary nuclear reactions and goes into detail about a simulation package designed to study them. The Chapter then goes on to describe how these reactions are measured on the NIF before showing the results of hundreds of said measurements. These measurements are sensitive to the fuel $\rho R$ of the implosion and are commonly used to infer it. Results are compared against other x-ray based measurement techniques and a few physics arguments are presented to explain observed discrepancies. 

\textbf{Chapter 3: A Neutron Recoil-Spectrometer for Measuring Yield and Determining Liner Areal Densities at the Z Facility}

This Chapter summarizes all of the work performed at the Z Facility. It describes the unique issues of trying to make neutron measurements on the Z facility. The concept of a neutron recoil-spectrometer is briefly introduced and proposed as a potential method for yield and liner $\rho R$ measurements. A proof of principle design was developed and the data taken is presented in this Chapter. The remainder of the Chapter builds off this starting design and proposes an improved shielded design. 

\textbf{Chapter 4: Measuring Stopping Power in Warm Dense Matter Plasmas at OMEGA}

This Chapter summarizes all of the work done on the OMEGA Laser Facility. The concepts of stopping power and Warm Dense Matter (WDM) are briefly introduced and connected to the $\rho R$ requirements of high gain ignition. An experimental platform to measure the stopping power of WDM plasmas using the OMEGA Laser Facility is thoroughly developed and discussed. A series of experiments using this platform were performed and the data is presented in this Chapter. 