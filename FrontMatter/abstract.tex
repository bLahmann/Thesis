\begin{abstractpage}
	
	This thesis summarizes three distinct projects that use the spectroscopy of fusion products to diagnose areal densities in Inertial Confinement Fusion (ICF) implosions or study stopping power to better understand the areal density requirements of said implosions. 
	
	At the Nation Ignition Facility (NIF), spectroscopy of secondary nuclear reactions on surrogate deuterium filled implosions are sensitive to hot spot areal density magnitude and asymmetry. Secondary DT neutrons are routinely measured on the NIF using four neutron time of flight (nToF) spectrometers positioned at different lines of sight. These measurements infer convergence ratios that differ from those inferred by x-ray imaging techniques. This discrepancy is explained by each method having different sensitivities to profiles and asymmetries. Additionally, the widths of the secondary DT neutron spectra are sensitive to mode-2 asymmetries and further confirm that these asymmetries are present in the hot-spot of NIF implosions.
	
	At the Z-facilty, spectroscopy of the primary DD neutron spectra from Magnetized Liner Inertial Fusion (MagLIF) implosions enables diagnosing yields and liner areal densities. Both of these quantities are extremely important for assessing implosion performance on the Z. Traditional nToF spectrometers face additional challenges at the Z facility due to large scattering sources from the machine and long neutron burn widths. For this reason, a CR-39 based neutron-recoil spectrometer has been developed for measuring the DD spectrum at the Z. A proof of principle design was fielded and the data is presented. A improved shielded design for accurately measuring the liner areal density is developed within.
	
	Finally, at the OMEGA laser facility, a unique experimental platform for accurately characterizing and measuring the stopping power of Warm Dense Matter (WDM) plasmas has been developed. The platform uses X-ray Thomson Scattering (XRTS) to characterize the plasma's temperature and ionization state. Proton spectroscopy is used to accurately measure the energy loss through the WDM subject. Results from several experiments indicate that WDM plasmas consistently have higher stopping than cold matter when sufficiently heated. 
	
\end{abstractpage}